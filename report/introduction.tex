The gap between processor speed and main memory speed has for many years increased dramatically. To
attack this problem several solutions has been proposed and implemented. One of them is prefetching
where the processor gets the instruction and data before it is used so it doesn't have to wait for
it when it actually has to use it. Prefetching helps to decrease the average memory access time
and thereby increases the overall computer efficiency.

Several schemes can be used to implement a prefetcher. Our prefetcher uses a scheme called
\emph{stride}. It essentially consists of a table that is index by the program counter (PC) and it
stores, amongst other things, a stride that is the difference between the two most recent miss
addresses for a particular program counter address. When the prefetcher detects a pattern in the
strides it uses the stride value to prefetch what it assumes to be data that the processor needs in
the near future.

