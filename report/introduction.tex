The gap between processor speed and main memory speed has for many years increased dramatically 
\cite{mem_cpu_gap}. To attack this problem several solutions has been proposed and implemented. One
of them is prefetching where the processor fetches instructions and data from memory in advance,
before it is used.  Prefetching attempts to decrease the average memory access time and thereby
increases the overall computer efficiency. More specifically it tries to reduce the miss rate in the
average memory access time equation: $ \textit{hit rate} \times \textit{hit time} + \textit{miss
rate} \times \textit{miss penalty} $.

%Several schemes can be used to implement a prefetcher. Our prefetcher uses a scheme called
%\emph{stride}. It essentially consists of a table that is index by the program counter (PC) and it
%stores all L2 cache access addresses. It then computes the difference between the two most recent
%memory accesses, called a stride, for a particular load instruction. When the prefetcher detects a
%pattern in the strides it uses the stride value to prefetch what it assumes to be data that the
%processor needs in the near future.

An important subclass of prefetchers is stride prefetchers. These try to find a pattern in the
difference between consecutive memory addresses (called a stride). If a pattern of strides is found
it can be used to prefetch the next address that will be needed. Our prefetcher is a stride
prefetcher.


