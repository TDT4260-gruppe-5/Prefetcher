As mentioned above our prefetcher is a stride prefetcher. It tries to detect a 
pattern in the access history, and use this pattern to predict future 
references. Patterns could be detected on a global or a local (per PC) basis.
After reviewing statistics gathered from SPEC CPU2000 benchmarks we decided to 
focus on local patterns. This was because the global access stream had little 
additional unique information that could not already be extracted from the 
local streams without using a lot of additional resources. In most benchmarks 
the memory accesses are done by a few instructions that operate in a very 
predictable fashion.

\subsection{Table}

\begin{figure}
	\begin{tikzpicture}
		\draw [->] (1,7.5) -- (2, 7.5);
		\draw (1.5, 7.8) node {PC};
		\foreach \x in {6,...,8}
			\draw (2, \x) rectangle (3, \x + 1);

		\draw [dotted] (2, 5.75) -- (2, 5.25);
		\draw [dotted] (3, 5.75) -- (3, 5.25);
		\draw (2, 5) -- (2, 5.25);
		\draw (2, 6) -- (2, 5.75);
		\draw (3, 5) -- (3, 5.25);
		\draw (3, 6) -- (3, 5.75);
		\draw (2, 4) rectangle  (3, 5);

		\foreach \x in {0,2}
			\draw (\x * 1.5 + 4, 6.75) rectangle  (\x * 1.5 + 5.5, 8.25);

		\draw [dashed] (3, 8) -- (4, 8.25);
		\draw [dashed] (3, 7) -- (4, 6.75);

		\draw [dotted] (5.5, 6.75) rectangle (7, 8.25);
		\draw (5.5, 6.75) -- (5.75, 6.75);
		\draw (6.75, 6.75) -- (7, 6.75);
		\draw (5.5, 8.25) -- (5.75, 8.25);
		\draw (6.75, 8.25) -- (7, 8.25);

		\draw (4.75, 7.5) node {$a_0$};
		\draw (7.75, 7.5) node {$a_{2n}$};
	\end{tikzpicture}
	\caption{Access History Table}
	\label{fig:table}
\end{figure}

To keep track of the previous memory accesses of a program counter we use a
table. Each table entry contains a list of the last memory accesses
and misses by the program counter that maps to this entry.
A graphical representation of the table can be seen in Fig.~\ref{fig:table}.
The exact number of accesses to store in each entry can be tweaked.
More addresses per entry allows larger patterns to be detected,
but reduces the number of entries using the same amount of memory.
Larger patterns might allow you to catch more, while more entries
reduces the chance of collision.

To save space we can store only the lower $k$ bits in every address. We can
also avoid saving the lower 6 bits because the simulated computer uses 64 byte
cachelines, therefore the lower 6 bits of addresses to blocks in the cache will
always be 0. With $k=16$ we store only 10 bits and can correctly identify
patterns as long as the biggest stride is less than $2^{16}$.

Because the table size is much smaller
than the number of possible program counters, several instructions could use
the same table entry. If this happened in a real program it could lead to less
than optimal prefetching. With a sufficiently large table the chance of two
instructions clashing would be negligible. In addition the table entry is found
by taking the program counter modulo the table size. This ensures that
instructions that is less than table size apart will map to different table
entries, meaning all instructions in a small loop is likely to map to different
table entries.

\subsection{Pattern Matching}

When the table begins to fill we are able to find patterns in the memory accesses.
To find these patterns we calculate relative and absolute strides.
The relative stride is the difference between two consecutive
addresses. Absolute strides are calculated by choosing a pattern size n,
and then computing the difference between the address at position
$n$ to every address before $n$, and then doing the same for the address
at position $2n$.

That is,
\[
	s_i = \begin{cases}
		a_i - a_n,    & \text{if } i  < n \\%\in \{0 \dots n -1\}\\
		a_i - a_{2n}, & \text{if } i \geq n % \{n \dots 2n - 1\}
	\end{cases}
\]
where $s_i$~is the $i$th absolute stride, and $a_i$~is the address at position
$i$ in the table. Note that the newest address is at index 0 in the table. A
perfect match is found if
\begin{equation}
\label{eq:match}
\forall i \in \{0 \dots n - 1\} \,.\, s_i = s_{i + n}
\end{equation}

It is also possible to find a partial match. We then simply relax the
requirement in equation \eqref{eq:match} that all strides have to match, to
that at least some percentage has to match.

An example of a perfect match can be seen in Table~\ref{table:pattern}.

%Partial matching is the reason for using absolute strides instead of the
%more intuitive relative stride. If we have a partial match, and use relative
%strides, there is no easy way to prefetch a

\begin{table}[htb]
	\caption{Example of pattern with n=4}
	\label{table:pattern}
	\centering
	\begin{tabular}{c|c|c|c|c}
		\bfseries Position &
		\bfseries Address &
		\bfseries Stride &
		\bfseries Stride &
		\bfseries Stride \\
		& &
		\bfseries (Relative) &
		\bfseries (Absolute) &
		\bfseries Index \\
		\hline
		0 & 896   & & \\
		  &	& 64 & 448 & 0 \\
		1 & 832  & & \\
		  & & 64 & 384 & 1 \\
		2 & 768 & & \\
		  & & 128 & 320 & 2 \\
		3 & 640 & & \\
		  & & 192 & 192 & 3 \\
		4 & 448 & & \\
		  &	& 64 & 448 & 4 \\
		5 & 384 & & \\
		  &	& 64 & 384 & 5 \\
		6 & 320 & & \\
		  &	& 128 & 320 & 6 \\
		7 & 192 & & \\
		  &	& 192 & 192 & 7 \\
		8 & 0 & & \\
	\end{tabular}
\end{table}

\subsection{Most Common Stride}

In addition to the pattern matching above, we also implemented most common stride
prefetching. This was added as a fallback measure if the pattern matching code
was unable to find a pattern.
It looks for the most common stride, and if the number
is more than some number
of these strides it will use this stride to prefetch a
set number of cache lines.
Table~\ref{table:mcs} shows a history of accesses
where the most common stride is 64.
\begin{table}
	\caption{Example of Most Common Stride}
	\centering
	\label{table:mcs}
	\begin{tabular}{c|c}
		\bfseries Address & \bfseries Stride (Relative)\\
		\hline
		0   & \\
		    & 64\\
		64  & \\
		    & 64\\
		128 & \\
		    & 128\\
		256 & \\
		    & 64\\
		320 & \\
		    & 64\\
		386 & \\
	\end{tabular}
\end{table}

\subsection{Algorithm}

Pseudocode for the algorithm used can be found in Algorithm~\ref{alg:stride}.

As we can see there are a few constants in the algorithm that can be
tuned. These are

\begin{tabular}{l p{4.5cm}}
	\bfseries THRESHOLD\subscript{stride} & Determines when there is a match.
	For perfect matching this should be equal to the pattern size\\
	\bfseries AGGR\subscript{stride} & The aggresiveness determines how far
	ahead we should prefetch \\
	\bfseries THRESHOLD\subscript{mcs} & \\
	\bfseries AGGR\subscript{mcs} &  Same as the stride equivalent \\
\end{tabular}

It is important to note that these ``constants'' are actually C macros in our
source code, so they could depend on any variables in scope on the location
they are used.

\algnotext{EndIf}
\algnotext{EndFor}
\algnotext{EndFunction}

\begin{algorithm}
	\caption{The prefetching algorithm}
	\label{alg:stride}
	\begin{algorithmic}
		\Function{prefetch}{$a_0$, $a_1$, \dots, $a_{2n}$}
		\If{$\neg$ \textsc{stride\_prefetch}($a_0$, $a_1$, \dots, $a_{2n}$)}
			\State \textsc{mcs\_prefetch}($a_0$, $a_1$, \dots, $a_{2n}$)
		\EndIf
		\EndFunction
		\\
		\Function{stride\_prefetching}{$a_0$, $a_1$, \dots, $a_{2n}$}
			\For{$i = n \to 1$}
				\State $hits \gets 0$
%				\State $hit_j \gets$ \textnormal{false} $\quad \text{for } \, 0 \leq j \leq i - 1$
				\For{$j = 0 \to i - 1$}
					\State $hit_j \gets$ \textnormal{false}
					\State $s_j \gets a_j - a_i$
					\State $s_{i + j} \gets a_{i + j} - a_{2i}$
				\EndFor
				\For{$j = 0 \to i - 1$}
					\If{$s_j \equiv s_{i + j}$}
						\State $hits \gets hits + 1$
						\State $hit_j \gets$ \textnormal{true}
					\EndIf
				\EndFor
				\If{$hits \geq$ THRESHOLD\subscript{stride}}
					\For{$j = 0 \to$ AGGR\subscript{stride}}
						\State $k \gets i - 1 - mod(i,j)$
						\If{$hit_k$}
							\State \textsc{prefetch}($a_0 + s_k \cdot (\lfloor \frac{j}{i} \rfloor + 1)$)
						\EndIf
					\EndFor
					\State {\bfseries return} \textnormal{true}
				\EndIf
			\EndFor

			\Return \textnormal{false}
		\EndFunction
		\\
		\Function{mcs\_prefetching}{$a_0$, $a_1$, \dots, $a_{2n}$}
			\State $mcs \gets s_0$
			\State $count \gets 0$

			\For{$i = 0 \to 2n$}
				\State $s_i \gets a_i - a_{i + 1}$
			\EndFor

			\For{$i = 0 \to 2n$}
				\State $c \gets 0$
				\State $m \gets s_i$
				\For{$j = i \to 2n$}
					\If{$s_i \equiv m$} \State $c \gets c + 1$ \EndIf
				\EndFor

				\If{$c > count$}
					\State $count \gets c$
					\State $mcs \gets m$
				\EndIf
			\EndFor
			\\
			\If{$count \geq$ THRESHOLD\subscript{mcs}}
				\For{$i = 1 \to$ AGGR\subscript{mcs}}
				\State \textsc{prefetch}($a_0 + mcs \cdot i$)
				\EndFor
			\EndIf
		\EndFunction
	\end{algorithmic}
\end{algorithm}
