As mentioned above our prefetcher is a stride prefetcher. It tries to detect a
pattern in the access history, and use this pattern to predict future
references. Patterns could be detected on a global or a local (per PC) basis.
After reviewing statistics gathered from SPEC CPU2000 benchmarks we decided to
focus on local patterns. This was because the global access stream had little
extra information that could not be extracted from the local streams with less
work. In most benchmarks the memory accesses are done by a few instructions
that operate in a very predictable fashion.

\subsection{Table}

To keep track of the previous memory accesses of a program counter we use a
table. Each table entry contains a list of the ten last memory accesses by the
program counter that maps to this entry. Because the table size is much smaller
than the number of possible program counters, several instructions could map to
the same table entry. If this happened in a real program it could lead to less
than optimal prefetching. With a sufficiently large table the chance of two
instructions clashing would be negligible. In addition the table entry is found
by taking the program counter modulo the table size. This ensures that
instructions that is less than table size apart will map to different table
entries.

\subsection{Pattern Matching}

The patterns our prefetcher tries to detect are on the form in
Figure~\ref{fig:pattern}. It looks at the n last memory accesses, and checks if
entry $k$ matches entry $k + \frac{n}{2}$. If the number of matching entries
are above some threshold, it will use the pattern to prefetch.

\subsection{Most Common Stride}

In addition to the pattern matching above, we also used most common stride
prefetching. This was added as a fallback measure if the pattern matching code
was unable to find a pattern. It looks for the most common stride, and if there
is more than some number of these strides it will use this stride to prefetch a
set number of cachelines. Table~\ref{table:mcs} shows a history of accesses
where the most common stride is 64.
\begin{table}
	\caption{Example of Most Common Stride}
	\label{table:mcs}
	\begin{tabular}{c|c}
		\bfseries Address & \bfseries Stride\\
		\hline
		0   & \\
		    & 64\\
		64  & \\
		    & 64\\
		128 & \\
		    & 128\\
		256 & \\
		    & 64\\
		320 & \\
		    & 64\\
		386 & \\
	\end{tabular}
\end{table}
