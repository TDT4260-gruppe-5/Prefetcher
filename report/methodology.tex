To evaluate the performance of various prefetchers, we have used
a subset of the SPEC CPU2000 benchmarks in association with the M5
simulator. The subset consists of bzip2\_graphic, bzip2\_program, bzip2\_source and twolf from
SPECint and ammp, applu, apsi, art110, art470, galgel, swim and wupwise from SPECfp.
These programs were chosen by the course staff for us.

The benchmarks were started at a checkpoint taken after
1,000,000,000 instructions, and then run for 100,000,000 instructions.
The simulated computer uses an ALPHA CPU running at 2GHz
with a 2-way 64KB L1 data cache. It also has an 8-way 2MB L2 cache that is
connected to main memory by an 8 byte wide bus running at
400MHz. Main memory has a latency of 30ns.

The algorithm has four parameters that can be changed. In addition there are
three parameters for the table: table size, the number of accesses to save per
entry and the number of bits per access. In total that gives seven parameters
that affect the performance. Since we had limited computing power there were no way we
could test all those, so we tested one or two at a time, with the rest fixed.
The order we tested them in were:

\begin{itemize}
	\item Most Common Stride parameters (aggressiveness and threshold)
	\item Table size
	\item Number of accesses per table entry
	\item Aggressiveness and prefetch distance for the pattern matching
\end{itemize}

When we tested the table parameters we did not try to stay within 8 KB of
memory, because that would require us to scale the number of entries and/or
bits per access at the same time.

\comment{
During testing of our prefetcher we tested changing many different parameters.

table size (n+1)
number of entries (pc)

Stride:
Threshold
aggressiveness

MCS:
Threshold
aggressiveness
}
