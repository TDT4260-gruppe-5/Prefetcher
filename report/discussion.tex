The prefetcher presented in this report ...

\subsection{Table Parameters}
At a certain point the effect of giving the prefetcher more memory is virtually
zero. Fig.~\ref{fig:table_size_chart} shows that there seems to be no point in
having more than 256 entries. This we believe is caused by instruction
conflicts in the table. When the table reaches a certain size, most
instructions get their own place in the table, and avoid being evicted by
conflicting instructions.
A interesting observation is that it dips down at 8192 entries. This seems
to be due to a single benchmark, \emph{twolf}.

Fig.~\ref{fig:bits} shows the speedup related to bits per access.
We see a great increase up to 12 bits, then it slowly flattens out,
and we get the same speedup for 18 and 20 bits.
This is expected, as more bits allows bigger stride patterns to fit.

The last table parameter we changed was the maximum pattern size.
This was done by changing the number of accesses per entry.
With $2n+1$ accesses we can match patterns up to size $n$.
These results can be seen in Fig.~\ref{fig:pattern_size}.
We also see that there is not much gain in having more than 13
accesses per entry (13 accesses gives us a max pattern size of 6).
However when we tested adding more accesses per entry, and thus
increasing the maximum pattern size, we had the aggressiveness
set to the pattern size. Therefore it is possible that the bigger
patterns lead to a too high aggressiveness. It would be interesting
to run the tests with static aggressiveness to see if we saw the
same fall in performance above 13 accesses
per entry.

If we take the best of all these parameters we end up with a table size
of 256, 13 accesses per entry and 18 bits per access. This makes
the total table $256 \cdot 13 \cdot 18 = 59904 \textrm{ bits} = 7488 \textrm{ bytes}$ large.

\subsection{Other parameters}
An interesting result is shown in Fig.~\ref{fig:mcs_tweaks}.
We see that most common stride works best when the aggressiveness
is as high as six, and the threshold is at three.

It was also interesting to see that partial matching did not fare very well.
With this in mind one could optimize the algorithm by using relative strides
instead of absolute strides and only do perfect matching.
